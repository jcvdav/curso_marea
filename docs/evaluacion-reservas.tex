\PassOptionsToPackage{unicode=true}{hyperref} % options for packages loaded elsewhere
\PassOptionsToPackage{hyphens}{url}
%
\documentclass[]{book}
\usepackage{lmodern}
\usepackage{amssymb,amsmath}
\usepackage{ifxetex,ifluatex}
\usepackage{fixltx2e} % provides \textsubscript
\ifnum 0\ifxetex 1\fi\ifluatex 1\fi=0 % if pdftex
  \usepackage[T1]{fontenc}
  \usepackage[utf8]{inputenc}
  \usepackage{textcomp} % provides euro and other symbols
\else % if luatex or xelatex
  \usepackage{unicode-math}
  \defaultfontfeatures{Ligatures=TeX,Scale=MatchLowercase}
\fi
% use upquote if available, for straight quotes in verbatim environments
\IfFileExists{upquote.sty}{\usepackage{upquote}}{}
% use microtype if available
\IfFileExists{microtype.sty}{%
\usepackage[]{microtype}
\UseMicrotypeSet[protrusion]{basicmath} % disable protrusion for tt fonts
}{}
\IfFileExists{parskip.sty}{%
\usepackage{parskip}
}{% else
\setlength{\parindent}{0pt}
\setlength{\parskip}{6pt plus 2pt minus 1pt}
}
\usepackage{hyperref}
\hypersetup{
            pdftitle={Costeo y Evaluación de Reservas Marinas},
            pdfauthor={Juan Carlos Villaseñor-Derbez},
            pdfborder={0 0 0},
            breaklinks=true}
\urlstyle{same}  % don't use monospace font for urls
\usepackage{longtable,booktabs}
% Fix footnotes in tables (requires footnote package)
\IfFileExists{footnote.sty}{\usepackage{footnote}\makesavenoteenv{longtable}}{}
\setlength{\emergencystretch}{3em}  % prevent overfull lines
\providecommand{\tightlist}{%
  \setlength{\itemsep}{0pt}\setlength{\parskip}{0pt}}
\setcounter{secnumdepth}{5}
% Redefines (sub)paragraphs to behave more like sections
\ifx\paragraph\undefined\else
\let\oldparagraph\paragraph
\renewcommand{\paragraph}[1]{\oldparagraph{#1}\mbox{}}
\fi
\ifx\subparagraph\undefined\else
\let\oldsubparagraph\subparagraph
\renewcommand{\subparagraph}[1]{\oldsubparagraph{#1}\mbox{}}
\fi

% set default figure placement to htbp
\makeatletter
\def\fps@figure{htbp}
\makeatother

\usepackage{booktabs}
\usepackage{longtable}
\usepackage[bf,singlelinecheck=off]{caption}

\setmainfont[UprightFeatures={SmallCapsFont=AlegreyaSC-Regular}]{Alegreya}

\usepackage{framed,color}
\definecolor{shadecolor}{RGB}{248,248,248}

\renewcommand{\textfraction}{0.05}
\renewcommand{\topfraction}{0.8}
\renewcommand{\bottomfraction}{0.8}
\renewcommand{\floatpagefraction}{0.75}

\renewenvironment{quote}{\begin{VF}}{\end{VF}}
\let\oldhref\href
\renewcommand{\href}[2]{#2\footnote{\url{#1}}}

\ifxetex
  \usepackage{letltxmacro}
  \setlength{\XeTeXLinkMargin}{1pt}
  \LetLtxMacro\SavedIncludeGraphics\includegraphics
  \def\includegraphics#1#{% #1 catches optional stuff (star/opt. arg.)
    \IncludeGraphicsAux{#1}%
  }%
  \newcommand*{\IncludeGraphicsAux}[2]{%
    \XeTeXLinkBox{%
      \SavedIncludeGraphics#1{#2}%
    }%
  }%
\fi

\makeatletter
\newenvironment{kframe}{%
\medskip{}
\setlength{\fboxsep}{.8em}
 \def\at@end@of@kframe{}%
 \ifinner\ifhmode%
  \def\at@end@of@kframe{\end{minipage}}%
  \begin{minipage}{\columnwidth}%
 \fi\fi%
 \def\FrameCommand##1{\hskip\@totalleftmargin \hskip-\fboxsep
 \colorbox{shadecolor}{##1}\hskip-\fboxsep
     % There is no \\@totalrightmargin, so:
     \hskip-\linewidth \hskip-\@totalleftmargin \hskip\columnwidth}%
 \MakeFramed {\advance\hsize-\width
   \@totalleftmargin\z@ \linewidth\hsize
   \@setminipage}}%
 {\par\unskip\endMakeFramed%
 \at@end@of@kframe}
\makeatother

\makeatletter
\@ifundefined{Shaded}{
}{\renewenvironment{Shaded}{\begin{kframe}}{\end{kframe}}}
\makeatother

\newenvironment{rmdblock}[1]
  {
  \begin{itemize}
  \renewcommand{\labelitemi}{
    \raisebox{-.7\height}[0pt][0pt]{
      {\setkeys{Gin}{width=3em,keepaspectratio}\includegraphics{images/#1}}
    }
  }
  \setlength{\fboxsep}{1em}
  \begin{kframe}
  \item
  }
  {
  \end{kframe}
  \end{itemize}
  }
\newenvironment{rmdnote}
  {\begin{rmdblock}{note}}
  {\end{rmdblock}}
\newenvironment{rmdcaution}
  {\begin{rmdblock}{caution}}
  {\end{rmdblock}}
\newenvironment{rmdimportant}
  {\begin{rmdblock}{important}}
  {\end{rmdblock}}
\newenvironment{rmdtip}
  {\begin{rmdblock}{tip}}
  {\end{rmdblock}}
\newenvironment{rmdwarning}
  {\begin{rmdblock}{warning}}
  {\end{rmdblock}}

\usepackage{makeidx}
\makeindex

\urlstyle{tt}

\usepackage{amsthm}
\makeatletter
\def\thm@space@setup{%
  \thm@preskip=8pt plus 2pt minus 4pt
  \thm@postskip=\thm@preskip
}
\makeatother

\frontmatter
\usepackage[]{natbib}
\bibliographystyle{apalike}

\title{Costeo y Evaluación de Reservas Marinas}
\author{Juan Carlos Villaseñor-Derbez}
\date{Bren School of Environmental Science \& Management, UCSB}

\begin{document}
\maketitle

{
\setcounter{tocdepth}{2}
\tableofcontents
}
\hypertarget{antes-de-empezar}{%
\chapter*{Antes de empezar}\label{antes-de-empezar}}


Este manual es la segunda iteración de los esfuerzos por impulsar el uso
de metodologías estandarizadas para la evaluación de reservas marinas.
Trabajos anteriores incluyen el manual generalizado de evaluación de
reservas marinas en México \citep{villaseorderbez_2017} y la publicación
arbitrada que presenta a
\href{https://turfeffect.shinyapps.io/marea/}{MAREA} como una
herramienta amigable y gratuita \citep{villasenorderbez_2018}. Esta
versión del manual pretende incorporar partes de ambos trabajos, pero
también incluye una serie de ejercicios prácticos para el uso de MAREA y
la nueva App de Costeo de Reservas. Además, el manual está públicamente
disponible en \href{https://jcvdav.github.io/curso_marea/}{internet},
donde el lector puede descargar el manual como PDF o EPUB para Kindle.

\hypertarget{requisitos}{%
\section{Requisitos}\label{requisitos}}

MAREA y la nueva App de Costeo de Reservas son aplicaciones web, y para
poder utilizarlas es necesario tener un explorador de internet y una
conexión estable. Aunque no siempre tenemos acceso a internet, este
formato nos evita problemas de compatibilidad entre diferentes sistemas
operativos. Si tienes un explorador de internet y una conexión estable,
puedes usar estas Apps.

Si participaste en uno de los cursos presenciales, el USB que recibise
contiene este manual como PDF y EPUB además de los
\href{https://github.com/jcvdav/curso_marea/materiales/datos}{datos
sintéticos} para los ejercicios prácticos y las
\href{https://github.com/jcvdav/curso_marea/materiales/diapositivas}{diapositivas
del curso}. Puedes distribuir libremente estos materiales, o
descargarlos desde el
\href{https://github.com/jcvdav/curso_marea}{repositorio de GitHub}. La
versión en línea siempre será la más actualizada.

\hypertarget{introduccion}{%
\chapter{Introducción}\label{introduccion}}

\hypertarget{antecedentes-en-la-evaluacion-de-reservas}{%
\chapter{Antecedentes en la evaluación de
reservas}\label{antecedentes-en-la-evaluacion-de-reservas}}

\hypertarget{dentro-vs.fuera}{%
\section{Dentro vs.~Fuera}\label{dentro-vs.fuera}}

\hypertarget{antes-vs.despues}{%
\section{Antes vs.~Después}\label{antes-vs.despues}}

\hypertarget{dentro-vs.fuera---antes-vs.despues}{%
\section{Dentro vs.~Fuera - Antes
vs.~Después}\label{dentro-vs.fuera---antes-vs.despues}}

\hypertarget{dentro-vs.fuera---antes-vs.despues-multiple}{%
\section{Dentro vs.~Fuera - Antes vs.~Después
multiple}\label{dentro-vs.fuera---antes-vs.despues-multiple}}

\hypertarget{evaluacion-de-reservas}{%
\chapter{Evaluación de reservas}\label{evaluacion-de-reservas}}

\hypertarget{objetivos-e-indicadores}{%
\section{Objetivos e indicadores}\label{objetivos-e-indicadores}}

\hypertarget{analisis-de-inferencia-de-causalidad}{%
\section{Análisis de inferencia de
causalidad}\label{analisis-de-inferencia-de-causalidad}}

\hypertarget{introduccion-a-marea}{%
\chapter{Introducción a MAREA}\label{introduccion-a-marea}}

\hypertarget{tipos-y-formatos-de-datos}{%
\section{Tipos y formatos de datos}\label{tipos-y-formatos-de-datos}}

\hypertarget{capacidades-y-limitaciones}{%
\section{Capacidades y limitaciones}\label{capacidades-y-limitaciones}}

\hypertarget{evaluacion-de-reservas-en-6-etapas}{%
\section{Evaluación de reservas en 6
etapas}\label{evaluacion-de-reservas-en-6-etapas}}

\hypertarget{interpretacion-de-resultados}{%
\section{Interpretación de
resultados}\label{interpretacion-de-resultados}}

\hypertarget{uso-de-marea}{%
\chapter{Uso de MAREA}\label{uso-de-marea}}

\hypertarget{evaluacion-de-indicadores-biologicos-para-1-reserva}{%
\section{Evaluación de indicadores biológicos para 1
reserva}\label{evaluacion-de-indicadores-biologicos-para-1-reserva}}

\hypertarget{evaluacion-de-indicadores-biologicos-y-especie-objetivo-para-1-reserva}{%
\section{Evaluación de indicadores biológicos y especie objetivo para 1
reserva}\label{evaluacion-de-indicadores-biologicos-y-especie-objetivo-para-1-reserva}}

\hypertarget{evaluacion-de-todos-los-indicadores-para-1-reserva}{%
\section{Evaluación de todos los indicadores para 1
reserva}\label{evaluacion-de-todos-los-indicadores-para-1-reserva}}

\hypertarget{evaluacion-de-todos-los-indicadores-para-varias-reservas-simultaneamente}{%
\section{Evaluación de todos los indicadores para varias reservas,
simultáneamente}\label{evaluacion-de-todos-los-indicadores-para-varias-reservas-simultaneamente}}

\hypertarget{errores-comunes-y-solucion-de-problemas}{%
\chapter{Errores comunes y solución de
problemas}\label{errores-comunes-y-solucion-de-problemas}}

\hypertarget{especie-indicador-no-tiene-diseno-baci}{%
\section{Especie / Indicador no tiene diseño
BACI}\label{especie-indicador-no-tiene-diseno-baci}}

\hypertarget{diferentes-especies-en-bases-biologicas-vs-pesca}{%
\section{Diferentes especies en bases biológicas vs
pesca}\label{diferentes-especies-en-bases-biologicas-vs-pesca}}

\bibliography{references.bib}

\backmatter
\printindex

\end{document}
