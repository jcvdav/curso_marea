\PassOptionsToPackage{unicode=true}{hyperref} % options for packages loaded elsewhere
\PassOptionsToPackage{hyphens}{url}
%
\documentclass[]{krantz}
\usepackage{lmodern}
\usepackage{amssymb,amsmath}
\usepackage{ifxetex,ifluatex}
\usepackage{fixltx2e} % provides \textsubscript
\ifnum 0\ifxetex 1\fi\ifluatex 1\fi=0 % if pdftex
  \usepackage[T1]{fontenc}
  \usepackage[utf8]{inputenc}
  \usepackage{textcomp} % provides euro and other symbols
\else % if luatex or xelatex
  \usepackage{unicode-math}
  \defaultfontfeatures{Ligatures=TeX,Scale=MatchLowercase}
\fi
% use upquote if available, for straight quotes in verbatim environments
\IfFileExists{upquote.sty}{\usepackage{upquote}}{}
% use microtype if available
\IfFileExists{microtype.sty}{%
\usepackage[]{microtype}
\UseMicrotypeSet[protrusion]{basicmath} % disable protrusion for tt fonts
}{}
\IfFileExists{parskip.sty}{%
\usepackage{parskip}
}{% else
\setlength{\parindent}{0pt}
\setlength{\parskip}{6pt plus 2pt minus 1pt}
}
\usepackage{hyperref}
\hypersetup{
            pdftitle={Costeo y Evaluación de Reservas Marinas con Shiny Apps},
            pdfauthor={Juan Carlos Villaseñor-Derbez},
            pdfborder={0 0 0},
            breaklinks=true}
\urlstyle{same}  % don't use monospace font for urls
\usepackage{color}
\usepackage{fancyvrb}
\newcommand{\VerbBar}{|}
\newcommand{\VERB}{\Verb[commandchars=\\\{\}]}
\DefineVerbatimEnvironment{Highlighting}{Verbatim}{commandchars=\\\{\}}
% Add ',fontsize=\small' for more characters per line
\usepackage{framed}
\definecolor{shadecolor}{RGB}{248,248,248}
\newenvironment{Shaded}{\begin{snugshade}}{\end{snugshade}}
\newcommand{\AlertTok}[1]{\textcolor[rgb]{0.94,0.16,0.16}{#1}}
\newcommand{\AnnotationTok}[1]{\textcolor[rgb]{0.56,0.35,0.01}{\textbf{\textit{#1}}}}
\newcommand{\AttributeTok}[1]{\textcolor[rgb]{0.77,0.63,0.00}{#1}}
\newcommand{\BaseNTok}[1]{\textcolor[rgb]{0.00,0.00,0.81}{#1}}
\newcommand{\BuiltInTok}[1]{#1}
\newcommand{\CharTok}[1]{\textcolor[rgb]{0.31,0.60,0.02}{#1}}
\newcommand{\CommentTok}[1]{\textcolor[rgb]{0.56,0.35,0.01}{\textit{#1}}}
\newcommand{\CommentVarTok}[1]{\textcolor[rgb]{0.56,0.35,0.01}{\textbf{\textit{#1}}}}
\newcommand{\ConstantTok}[1]{\textcolor[rgb]{0.00,0.00,0.00}{#1}}
\newcommand{\ControlFlowTok}[1]{\textcolor[rgb]{0.13,0.29,0.53}{\textbf{#1}}}
\newcommand{\DataTypeTok}[1]{\textcolor[rgb]{0.13,0.29,0.53}{#1}}
\newcommand{\DecValTok}[1]{\textcolor[rgb]{0.00,0.00,0.81}{#1}}
\newcommand{\DocumentationTok}[1]{\textcolor[rgb]{0.56,0.35,0.01}{\textbf{\textit{#1}}}}
\newcommand{\ErrorTok}[1]{\textcolor[rgb]{0.64,0.00,0.00}{\textbf{#1}}}
\newcommand{\ExtensionTok}[1]{#1}
\newcommand{\FloatTok}[1]{\textcolor[rgb]{0.00,0.00,0.81}{#1}}
\newcommand{\FunctionTok}[1]{\textcolor[rgb]{0.00,0.00,0.00}{#1}}
\newcommand{\ImportTok}[1]{#1}
\newcommand{\InformationTok}[1]{\textcolor[rgb]{0.56,0.35,0.01}{\textbf{\textit{#1}}}}
\newcommand{\KeywordTok}[1]{\textcolor[rgb]{0.13,0.29,0.53}{\textbf{#1}}}
\newcommand{\NormalTok}[1]{#1}
\newcommand{\OperatorTok}[1]{\textcolor[rgb]{0.81,0.36,0.00}{\textbf{#1}}}
\newcommand{\OtherTok}[1]{\textcolor[rgb]{0.56,0.35,0.01}{#1}}
\newcommand{\PreprocessorTok}[1]{\textcolor[rgb]{0.56,0.35,0.01}{\textit{#1}}}
\newcommand{\RegionMarkerTok}[1]{#1}
\newcommand{\SpecialCharTok}[1]{\textcolor[rgb]{0.00,0.00,0.00}{#1}}
\newcommand{\SpecialStringTok}[1]{\textcolor[rgb]{0.31,0.60,0.02}{#1}}
\newcommand{\StringTok}[1]{\textcolor[rgb]{0.31,0.60,0.02}{#1}}
\newcommand{\VariableTok}[1]{\textcolor[rgb]{0.00,0.00,0.00}{#1}}
\newcommand{\VerbatimStringTok}[1]{\textcolor[rgb]{0.31,0.60,0.02}{#1}}
\newcommand{\WarningTok}[1]{\textcolor[rgb]{0.56,0.35,0.01}{\textbf{\textit{#1}}}}
\usepackage{longtable,booktabs}
% Fix footnotes in tables (requires footnote package)
\IfFileExists{footnote.sty}{\usepackage{footnote}\makesavenoteenv{longtable}}{}
\usepackage{graphicx,grffile}
\makeatletter
\def\maxwidth{\ifdim\Gin@nat@width>\linewidth\linewidth\else\Gin@nat@width\fi}
\def\maxheight{\ifdim\Gin@nat@height>\textheight\textheight\else\Gin@nat@height\fi}
\makeatother
% Scale images if necessary, so that they will not overflow the page
% margins by default, and it is still possible to overwrite the defaults
% using explicit options in \includegraphics[width, height, ...]{}
\setkeys{Gin}{width=\maxwidth,height=\maxheight,keepaspectratio}
\setlength{\emergencystretch}{3em}  % prevent overfull lines
\providecommand{\tightlist}{%
  \setlength{\itemsep}{0pt}\setlength{\parskip}{0pt}}
\setcounter{secnumdepth}{5}
% Redefines (sub)paragraphs to behave more like sections
\ifx\paragraph\undefined\else
\let\oldparagraph\paragraph
\renewcommand{\paragraph}[1]{\oldparagraph{#1}\mbox{}}
\fi
\ifx\subparagraph\undefined\else
\let\oldsubparagraph\subparagraph
\renewcommand{\subparagraph}[1]{\oldsubparagraph{#1}\mbox{}}
\fi

% set default figure placement to htbp
\makeatletter
\def\fps@figure{htbp}
\makeatother

\usepackage{booktabs}
\usepackage{longtable}
\usepackage[bf,singlelinecheck=off]{caption}

\usepackage{framed,color}
\definecolor{shadecolor}{RGB}{248,248,248}

\renewcommand{\textfraction}{0.05}
\renewcommand{\topfraction}{0.8}
\renewcommand{\bottomfraction}{0.8}
\renewcommand{\floatpagefraction}{0.75}

\renewenvironment{quote}{\begin{VF}}{\end{VF}}
\let\oldhref\href
\renewcommand{\href}[2]{#2\footnote{\url{#1}}}

\makeatletter
\newenvironment{kframe}{%
\medskip{}
\setlength{\fboxsep}{.8em}
 \def\at@end@of@kframe{}%
 \ifinner\ifhmode%
  \def\at@end@of@kframe{\end{minipage}}%
  \begin{minipage}{\columnwidth}%
 \fi\fi%
 \def\FrameCommand##1{\hskip\@totalleftmargin \hskip-\fboxsep
 \colorbox{shadecolor}{##1}\hskip-\fboxsep
     % There is no \\@totalrightmargin, so:
     \hskip-\linewidth \hskip-\@totalleftmargin \hskip\columnwidth}%
 \MakeFramed {\advance\hsize-\width
   \@totalleftmargin\z@ \linewidth\hsize
   \@setminipage}}%
 {\par\unskip\endMakeFramed%
 \at@end@of@kframe}
\makeatother

\renewenvironment{Shaded}{\begin{kframe}}{\end{kframe}}

\usepackage{makeidx}
\makeindex

\urlstyle{tt}

\usepackage{amsthm}
\makeatletter
\def\thm@space@setup{%
  \thm@preskip=8pt plus 2pt minus 4pt
  \thm@postskip=\thm@preskip
}
\makeatother

\frontmatter
\usepackage[]{natbib}
\bibliographystyle{apalike}

\title{Costeo y Evaluación de Reservas Marinas con Shiny Apps}
\providecommand{\subtitle}[1]{}
\subtitle{Una guía para gestores ambientales}
\author{Juan Carlos Villaseñor-Derbez}
\date{Bren School of Environmental Science \& Management, UCSB}

\begin{document}
\maketitle

{
\setcounter{tocdepth}{2}
\tableofcontents
}
\hypertarget{antes-de-empezar}{%
\chapter*{Antes de empezar}\label{antes-de-empezar}}


Este manual es la segunda iteración de los esfuerzos por impulsar el uso
de metodologías estandarizadas para la evaluación de reservas marinas.
Trabajos anteriores incluyen el manual generalizado de evaluación de
reservas marinas en México \citep{villaseorderbez_2017} y la publicación
arbitrada que presenta a
\href{https://turfeffect.shinyapps.io/marea/}{MAREA} como una
herramienta amigable y gratuita \citep{villasenorderbez_2018}. Esta
versión del manual pretende incorporar partes de ambos trabajos, pero
también incluye una serie de ejercicios prácticos para el uso de MAREA y
la nueva App de Costeo de Reservas. Además, el manual está públicamente
disponible en \href{https://jcvdav.github.io/curso_marea/}{internet},
donde el lector puede descargar el manual como PDF o EPUB para Kindle.

Aunque el manual y la Aplicación para Evaluación de Reservas Marinas
(MAREA) pueden ser utilizados alrededor del mundo, es importante
mencionar que el proyecto fue diseñado para evaluar la efectividad de
las reservas marinas en México. Por lo tanto, las metodologías
utilizadas reflejan las necesidades de las comunidades costeras
mexicanas, y no debe de interpretarse como un conjunto de instrucciones
definitivas. Aún así, creemos que la guía ha sido creada para permitir
su aplicación en otros lugares con el mismo fin.

\hypertarget{requisitos}{%
\section{Requisitos}\label{requisitos}}

MAREA y la nueva App de Costeo de Reservas son aplicaciones web, y para
poder utilizarlas es necesario tener un explorador de internet y una
conexión estable. Aunque no siempre tenemos acceso a internet, este
formato nos evita problemas de compatibilidad entre diferentes sistemas
operativos. Si tienes un explorador de internet y una conexión estable,
puedes usar estas Apps.

Si participaste en uno de los cursos presenciales, el USB que recibise
contiene este manual como PDF y EPUB además de los
\href{https://github.com/jcvdav/curso_marea/materiales/datos}{datos
sintéticos} para los ejercicios prácticos y las
\href{https://github.com/jcvdav/curso_marea/materiales/diapositivas}{diapositivas
del curso}. Puedes distribuir libremente estos materiales, o
descargarlos desde el
\href{https://github.com/jcvdav/curso_marea}{repositorio de GitHub}. La
versión en línea siempre será la más actualizada.

\clearpage

\hypertarget{part-parte-i}{%
\part{Parte I}\label{part-parte-i}}

\hypertarget{introduccion}{%
\chapter{Introducción}\label{introduccion}}

El desarrollo de \href{https://turfeffect.shinyapps.io/marea/}{MAREA}
fue motivado por la necesidad de proveer metodologías estandarizadas y
rigurosas para evaluar las zonas de refugio pesquero, un tipo de
reservass marinas diseñadas como herramientas de manejo pesuero
\citep{nom}.

Comunidad y Biodiversidad (COBI) es una Organización de la Sociedad
Civil (OSC) involucrada en el diseño, implementación y manejo de
reservas marinas en el Caribe Mexicano, así como el Golfo de California
y la costa Pacífica de Baja California.

\hypertarget{formas-de-evaluar-reservas}{%
\chapter{Formas de evaluar reservas}\label{formas-de-evaluar-reservas}}

\hypertarget{dentro-vs.fuera}{%
\section{Dentro vs.~Fuera}\label{dentro-vs.fuera}}

\hypertarget{antes-vs.despues}{%
\section{Antes vs.~Después}\label{antes-vs.despues}}

\hypertarget{dentro-vs.fuera---antes-vs.despues}{%
\section{Dentro vs.~Fuera - Antes
vs.~Después}\label{dentro-vs.fuera---antes-vs.despues}}

\hypertarget{dentro-vs.fuera---antes-vs.despues-multiple}{%
\section{Dentro vs.~Fuera - Antes vs.~Después
multiple}\label{dentro-vs.fuera---antes-vs.despues-multiple}}

\hypertarget{evaluacion-de-reservas}{%
\chapter{Evaluación de reservas}\label{evaluacion-de-reservas}}

\hypertarget{objetivos-e-indicadores}{%
\section{Objetivos e indicadores}\label{objetivos-e-indicadores}}

\hypertarget{analisis-de-inferencia-de-causalidad}{%
\section{Análisis de inferencia de
causalidad}\label{analisis-de-inferencia-de-causalidad}}

\hypertarget{introduccion-a-marea}{%
\chapter{Introducción a MAREA}\label{introduccion-a-marea}}

\hypertarget{tipos-y-formatos-de-datos}{%
\section{Tipos y formatos de datos}\label{tipos-y-formatos-de-datos}}

\hypertarget{capacidades-y-limitaciones}{%
\section{Capacidades y limitaciones}\label{capacidades-y-limitaciones}}

\hypertarget{evaluacion-de-reservas-en-6-etapas}{%
\section{Evaluación de reservas en 6
etapas}\label{evaluacion-de-reservas-en-6-etapas}}

\hypertarget{interpretacion-de-resultados}{%
\section{Interpretación de
resultados}\label{interpretacion-de-resultados}}

\hypertarget{part-parte-ii}{%
\part{Parte II}\label{part-parte-ii}}

\hypertarget{uso-de-marea}{%
\chapter{Uso de MAREA}\label{uso-de-marea}}

\hypertarget{evaluacion-de-indicadores-biologicos-para-1-reserva}{%
\section{Evaluación de indicadores biológicos para 1
reserva}\label{evaluacion-de-indicadores-biologicos-para-1-reserva}}

\hypertarget{evaluacion-de-indicadores-biologicos-y-especie-objetivo-para-1-reserva}{%
\section{Evaluación de indicadores biológicos y especie objetivo para 1
reserva}\label{evaluacion-de-indicadores-biologicos-y-especie-objetivo-para-1-reserva}}

\hypertarget{evaluacion-de-todos-los-indicadores-para-1-reserva}{%
\section{Evaluación de todos los indicadores para 1
reserva}\label{evaluacion-de-todos-los-indicadores-para-1-reserva}}

\hypertarget{evaluacion-de-todos-los-indicadores-para-varias-reservas-simultaneamente}{%
\section{Evaluación de todos los indicadores para varias reservas,
simultáneamente}\label{evaluacion-de-todos-los-indicadores-para-varias-reservas-simultaneamente}}

\hypertarget{errores-comunes-y-solucion-de-problemas}{%
\chapter{Errores comunes y solución de
problemas}\label{errores-comunes-y-solucion-de-problemas}}

\hypertarget{especie-indicador-no-tiene-diseno-baci}{%
\section{Especie / Indicador no tiene diseño
BACI}\label{especie-indicador-no-tiene-diseno-baci}}

\hypertarget{diferentes-especies-en-bases-biologicas-vs-pesca}{%
\section{Diferentes especies en bases biológicas vs
pesca}\label{diferentes-especies-en-bases-biologicas-vs-pesca}}

\hypertarget{appendix-appendice}{%
\appendix \addcontentsline{toc}{chapter}{\appendixname}}


\hypertarget{datos-sinteticos}{%
\chapter{Datos sintéticos}\label{datos-sinteticos}}

Este apéndice muestra el código usado para obtener los datos sintéticos
del curso y el manual de evaluación. Primero, debemos definir una serie
de variables que contengan los valores predeterminados o rangos de
valores que cada variable puede tomar. Para los propósitos del curso,
generaremos únicamente información biológica y económica de peces.

\begin{Shaded}
\begin{Highlighting}[]
\CommentTok{# Cargamos los paquetes que necesitamos}
\KeywordTok{suppressPackageStartupMessages}\NormalTok{(\{}
  \KeywordTok{library}\NormalTok{(magrittr)}
  \KeywordTok{library}\NormalTok{(tidyverse)}
\NormalTok{\})}
\end{Highlighting}
\end{Shaded}

\begin{Shaded}
\begin{Highlighting}[]
\CommentTok{######################################}
\CommentTok{# Generar variables predeterminadas}
\CommentTok{######################################}

\CommentTok{# Las fechas estaran centradas en el dia 1 de cada mes}
\NormalTok{dia <-}\StringTok{ }\DecValTok{1}

\CommentTok{# Los muestreos ocurren aleatoriamente entre abril y junio}
\NormalTok{mes <-}\StringTok{ }\DecValTok{4}\OperatorTok{:}\DecValTok{6}

\CommentTok{# Generaremos datos del 200 al 2018}
\NormalTok{ano <-}\StringTok{ }\DecValTok{2000}\OperatorTok{:}\DecValTok{2018}

\CommentTok{# El estado va a ser NA}
\NormalTok{estado <-}\StringTok{ }\OtherTok{NA}

\CommentTok{# La comunidad imaginaria va a ser Las Positas}
\NormalTok{comunidad <-}\StringTok{ "Las Positas"}

\CommentTok{# En Las Positas hay 4 sitios, dos reservas y dos controles}
\CommentTok{# el tipo de sitio se define mas adelante}
\NormalTok{sitio <-}\StringTok{ }\KeywordTok{c}\NormalTok{(}\StringTok{"Las cruces"}\NormalTok{,}
           \StringTok{"Cerro prieto"}\NormalTok{,}
           \StringTok{"Calencho"}\NormalTok{,}
           \StringTok{"Popotla"}\NormalTok{)}

\CommentTok{# No es necesario definir el habitat}
\NormalTok{habitat <-}\StringTok{ }\OtherTok{NA}

\CommentTok{# La zona es determinada con esta funcion}
\NormalTok{zona <-}\StringTok{ }\ControlFlowTok{function}\NormalTok{(sitio)\{}
  \KeywordTok{ifelse}\NormalTok{(sitio }\OperatorTok\StringTok{ }\KeywordTok{c}\NormalTok{(}\StringTok{"Las cruces"}\NormalTok{,}
                      \StringTok{"Cerro prieto"}\NormalTok{),}
         \StringTok{"Reserva"}\NormalTok{,}
         \StringTok{"Control"}\NormalTok{)}
\NormalTok{\}}

\CommentTok{# Tipo de proteccion es NA}
\NormalTok{tipo_proteccion <-}\StringTok{ }\OtherTok{NA}

\CommentTok{# ANP es NA}
\NormalTok{ANP <-}\StringTok{ }\OtherTok{NA}

\CommentTok{# Lista de posibles buzos monitores (http://www.laff.bren.ucsb.edu/laff-network/alumni)}
\NormalTok{buzo_monitor <-}\StringTok{ }\KeywordTok{c}\NormalTok{(}\StringTok{"Caio Faro"}\NormalTok{,}
                  \StringTok{"Alexandra Smith"}\NormalTok{,}
                  \StringTok{"Diana Flores"}\NormalTok{, }
                  \StringTok{"Ignacia Rivera"}\NormalTok{,}
                  \StringTok{"Wagner Quiros"}\NormalTok{,}
                  \StringTok{"Gonzalo Banda"}\NormalTok{,}
                  \StringTok{"Camila Vargas"}\NormalTok{,}
                  \StringTok{"Diego Undurraga"}\NormalTok{,}
                  \StringTok{"Denise Garcia"}\NormalTok{,}
                  \StringTok{"Cristobal Libertad"}\NormalTok{,}
                  \StringTok{"Catalina Milagros"}\NormalTok{)}

\CommentTok{# Horas iniciales arbitrarias}
\NormalTok{hora_inicial <-}\StringTok{ }\KeywordTok{c}\NormalTok{(}\StringTok{"6:50"}\NormalTok{, }\StringTok{"8:40"}\NormalTok{, }\StringTok{"10:20"}\NormalTok{, }\StringTok{"12:15"}\NormalTok{, }\StringTok{"13:40"}\NormalTok{, }\StringTok{"14:45"}\NormalTok{, }\StringTok{"15:20"}\NormalTok{)}

\CommentTok{# Rango de profundidades iniciales posibles}
\NormalTok{profundidad_inicial <-}\StringTok{ }\DecValTok{5}\OperatorTok{:}\DecValTok{27}

\CommentTok{# Esta funcion inventa una profundidad final}
\CommentTok{# segun la profundidad inicial}
\NormalTok{profundidad_final <-}\StringTok{ }\ControlFlowTok{function}\NormalTok{(profundidad_inicial)\{}
  \KeywordTok{round}\NormalTok{(profundidad_inicial }\OperatorTok{+}\StringTok{ }\KeywordTok{rnorm}\NormalTok{(}\DataTypeTok{n =} \DecValTok{1}\NormalTok{, }\DataTypeTok{mean =} \DecValTok{0}\NormalTok{, }\DataTypeTok{sd =} \DecValTok{1}\NormalTok{), }\DataTypeTok{digits =} \DecValTok{1}\NormalTok{)}
\NormalTok{\}}

\CommentTok{# Rango de temperaturas}
\NormalTok{temperatura <-}\StringTok{ }\DecValTok{25}\OperatorTok{:}\DecValTok{27}

\CommentTok{# Rango de visibilidades}
\NormalTok{visibilidad <-}\StringTok{ }\DecValTok{3}\OperatorTok{:}\DecValTok{12}

\CommentTok{# Corriente es NA}
\NormalTok{corriente <-}\StringTok{ }\OtherTok{NA}

\CommentTok{# Numeros de transectos}
\NormalTok{transecto <-}\StringTok{ }\DecValTok{1}\OperatorTok{:}\DecValTok{12}

\CommentTok{# Crear un origen en comun para las secuencias aleatorias}
\KeywordTok{set.seed}\NormalTok{(}\DecValTok{42}\NormalTok{)}

\CommentTok{# De la lista de especies filtramos para tener}
\CommentTok{# especies menores a 160 cm y que tengan todos}
\CommentTok{# los parametros de a,b, NT y Lmax}
\NormalTok{spp <-}\StringTok{ }\NormalTok{MPAtools}\OperatorTok{::}\NormalTok{species_bio }\OperatorTok
\StringTok{  }\KeywordTok{filter}\NormalTok{(Lmax }\OperatorTok{<}\StringTok{ }\DecValTok{160}\NormalTok{) }\OperatorTok\StringTok{ }
\StringTok{  }\KeywordTok{select}\NormalTok{(GeneroEspecie, a, b, NT, Lmax) }\OperatorTok
\StringTok{  }\KeywordTok{drop_na}\NormalTok{() }\OperatorTok\StringTok{ }
\StringTok{  }\KeywordTok{sample_n}\NormalTok{(}\DecValTok{15}\NormalTok{)}

\CommentTok{# Crear un vector con todas las especies}
\NormalTok{genero_especie <-}\StringTok{ }\NormalTok{spp}\OperatorTok{$}\NormalTok{GeneroEspecie}

\CommentTok{# Esta funcion inventa una talla observada con una }
\CommentTok{# distribucion normal con promedio = la mitad entre}
\CommentTok{# 0 y la longitud maxima reportada y desviacion}
\CommentTok{# estandar = 0.3 * el promedio}
\NormalTok{tallas <-}\StringTok{ }\ControlFlowTok{function}\NormalTok{(spp, generoespecie)\{}
  
  \CommentTok{# calcular talla media}
\NormalTok{  talla <-}\StringTok{ }\NormalTok{spp }\OperatorTok\StringTok{ }
\StringTok{    }\KeywordTok{filter}\NormalTok{(GeneroEspecie }\OperatorTok{==}\StringTok{ }\NormalTok{generoespecie) }\OperatorTok\StringTok{ }
\StringTok{    }\NormalTok{Lmax }\OperatorTok{/}\StringTok{ }\DecValTok{2}
  
  \CommentTok{# obtener ruido al rededor de la talla media}
\NormalTok{  noise <-}\StringTok{ }\KeywordTok{rnorm}\NormalTok{(}\DataTypeTok{n =} \DecValTok{1}\NormalTok{, }\DataTypeTok{mean =} \DecValTok{0}\NormalTok{, }\DataTypeTok{sd =} \FloatTok{0.3} \OperatorTok{*}\StringTok{ }\NormalTok{talla }\OperatorTok{/}\StringTok{ }\DecValTok{2}\NormalTok{)}
  
  \CommentTok{# Redondear para evitar decimales}
  \KeywordTok{round}\NormalTok{(talla }\OperatorTok{+}\StringTok{ }\NormalTok{noise)}
\NormalTok{\}}

\CommentTok{# Esta funcion regresa la abundancia de la especie}
\CommentTok{# que es un numero que sigue una distribucion de}
\CommentTok{# poisson con Lambda = 12}
\NormalTok{mean_sp <-}\StringTok{ }\ControlFlowTok{function}\NormalTok{(generoespecie)\{}
  \KeywordTok{rpois}\NormalTok{(}\DataTypeTok{n =} \DecValTok{1}\NormalTok{, }\DataTypeTok{lambda =} \DecValTok{12}\NormalTok{)}
\NormalTok{\}}

\CommentTok{# Esta funcion regresa el par RC para cada sitio}
\NormalTok{rc <-}\StringTok{ }\ControlFlowTok{function}\NormalTok{(sitio)\{}
  \KeywordTok{ifelse}\NormalTok{(sitio }\OperatorTok\StringTok{ }\KeywordTok{c}\NormalTok{(}\StringTok{"Las cruces"}\NormalTok{,}
                      \StringTok{"Calencho"}\NormalTok{),}
         \StringTok{"Las cruces - Calencho"}\NormalTok{,}
         \StringTok{"Cerro prieto - Popotla"}\NormalTok{)}
\NormalTok{\}}
\end{Highlighting}
\end{Shaded}

\begin{Shaded}
\begin{Highlighting}[]
\CommentTok{######################################}
\CommentTok{# Simular datos}
\CommentTok{######################################}

\CommentTok{# Crear un data.frame vacio}
\NormalTok{datos <-}\StringTok{ }\KeywordTok{tibble}\NormalTok{(}\DataTypeTok{Dia =} \OtherTok{NA}\NormalTok{,}
                \DataTypeTok{Mes =} \OtherTok{NA}\NormalTok{,}
                \DataTypeTok{Ano =} \OtherTok{NA}\NormalTok{,}
                \DataTypeTok{Estado =} \OtherTok{NA}\NormalTok{,}
                \DataTypeTok{Comunidad =} \OtherTok{NA}\NormalTok{,}
                \DataTypeTok{Sitio =} \OtherTok{NA}\NormalTok{,}
                \DataTypeTok{Latitud =} \OtherTok{NA}\NormalTok{,}
                \DataTypeTok{Longitud =} \OtherTok{NA}\NormalTok{,}
                \DataTypeTok{Habitat =} \OtherTok{NA}\NormalTok{,}
                \DataTypeTok{Zona =} \OtherTok{NA}\NormalTok{,}
                \DataTypeTok{TipoProteccion =} \OtherTok{NA}\NormalTok{,}
                \DataTypeTok{ANP =} \OtherTok{NA}\NormalTok{,}
                \DataTypeTok{BuzoMonitor =} \OtherTok{NA}\NormalTok{,}
                \DataTypeTok{HoraInicial =} \OtherTok{NA}\NormalTok{,}
                \DataTypeTok{ProfundidadInicial =} \OtherTok{NA}\NormalTok{,}
                \DataTypeTok{ProfundidadFinal =} \OtherTok{NA}\NormalTok{,}
                \DataTypeTok{Temperatura =} \OtherTok{NA}\NormalTok{,}
                \DataTypeTok{Visibilidad =} \OtherTok{NA}\NormalTok{,}
                \DataTypeTok{Corriente =} \OtherTok{NA}\NormalTok{,}
                \DataTypeTok{Transecto =} \OtherTok{NA}\NormalTok{,}
                \DataTypeTok{Genero =} \OtherTok{NA}\NormalTok{,}
                \DataTypeTok{Especie =} \OtherTok{NA}\NormalTok{,}
                \DataTypeTok{GeneroEspecie =} \OtherTok{NA}\NormalTok{,}
                \DataTypeTok{Sexo =} \OtherTok{NA}\NormalTok{,}
                \DataTypeTok{Talla =} \OtherTok{NA}\NormalTok{,}
                \DataTypeTok{ClaseTalla =} \OtherTok{NA}\NormalTok{,}
                \DataTypeTok{Abundancia =} \OtherTok{NA}\NormalTok{,}
                \DataTypeTok{RC =} \OtherTok{NA}\NormalTok{)}

\CommentTok{# Definir un ciclo para iterar cada año}
\ControlFlowTok{for}\NormalTok{(i }\ControlFlowTok{in}\NormalTok{ ano)\{}
  \CommentTok{# El ano es determinado por el ciclo}
\NormalTok{  Ano <-}\StringTok{ }\NormalTok{i}
  
  \CommentTok{# El estado es constante}
\NormalTok{  Estado <-}\StringTok{ }\NormalTok{estado}
  
  \CommentTok{# La comunidad es constante}
\NormalTok{  Comunidad <-}\StringTok{ }\NormalTok{comunidad}
  
  \CommentTok{# Definir un ciclo para iterar cada sitio}
  \ControlFlowTok{for}\NormalTok{(j }\ControlFlowTok{in}\NormalTok{ sitio)\{}
    
    \CommentTok{# El sitio es determinado por el ciclo}
\NormalTok{    Sitio <-}\StringTok{ }\NormalTok{j}
    
    \CommentTok{#La latitud y longitud son NAs}
\NormalTok{    Latitud <-}\StringTok{ }\OtherTok{NA}
\NormalTok{    Longitud <-}\StringTok{ }\OtherTok{NA}
    
    \CommentTok{# El habitat es constante (NA)}
\NormalTok{    Habitat <-}\StringTok{ }\NormalTok{habitat}
    
    \CommentTok{# Definir la zona segun la funcion anterior}
\NormalTok{    Zona <-}\StringTok{ }\KeywordTok{zona}\NormalTok{(j)}
    
    \CommentTok{# El tipo de proteccion es constante (NA)}
\NormalTok{    TipoProteccion <-}\StringTok{ }\NormalTok{tipo_proteccion}
    
    \CommentTok{# El ANP es constante (NA)}
\NormalTok{    ANP <-}\StringTok{ }\NormalTok{ANP}
    
    \CommentTok{# Definir un ciclo para iterar cada transecto}
    \ControlFlowTok{for}\NormalTok{(k }\ControlFlowTok{in}\NormalTok{ transecto)\{}
      
\NormalTok{      Dia <-}\StringTok{ }\NormalTok{dia}
      
      \CommentTok{# Aleatoriamente muestreamos un mes de la lista anterior (mes)}
\NormalTok{      Mes <-}\StringTok{ }\KeywordTok{sample}\NormalTok{(}\DataTypeTok{x =}\NormalTok{ mes,}
                    \DataTypeTok{size =}\NormalTok{ 1L)}
      
      \CommentTok{# Escoger aleatoriamente un buzo monitor}
\NormalTok{      BuzoMonitor <-}\StringTok{ }\KeywordTok{sample}\NormalTok{(}\DataTypeTok{x =}\NormalTok{ buzo_monitor,}
                            \DataTypeTok{size =}\NormalTok{ 1L)}
      
      \CommentTok{# Escoger aleatoriamente la hora inicial}
\NormalTok{      HoraInicial <-}\StringTok{ }\NormalTok{hora_inicial[}\KeywordTok{sample}\NormalTok{(}\DataTypeTok{x =} \DecValTok{1}\OperatorTok{:}\DecValTok{7}\NormalTok{,}
                                         \DataTypeTok{size =}\NormalTok{ 1L)]}
      
      \CommentTok{# Escoger alteatoriamente la profundidad inicial}
\NormalTok{      ProfundidadInicial <-}\StringTok{ }\KeywordTok{sample}\NormalTok{(}\DataTypeTok{x =}\NormalTok{ profundidad_inicial,}
                                   \DataTypeTok{size =}\NormalTok{ 1L)}
      
      \CommentTok{# Calcular la profundidad final segun la funcion anterior}
\NormalTok{      ProfundidadFinal <-}\StringTok{ }\KeywordTok{profundidad_final}\NormalTok{(ProfundidadInicial)}
      
      \CommentTok{# Escoger una temperatura alteatoria}
\NormalTok{      Temperatura <-}\StringTok{ }\KeywordTok{sample}\NormalTok{(}\DataTypeTok{x =}\NormalTok{ temperatura,}
                            \DataTypeTok{size =}\NormalTok{ 1L)}
      
      \CommentTok{# Escoger una visibilidad aleatoria}
\NormalTok{      Visibilidad <-}\StringTok{ }\KeywordTok{sample}\NormalTok{(}\DataTypeTok{x =}\NormalTok{ visibilidad,}
                            \DataTypeTok{size =}\NormalTok{ 1L)}
      
      \CommentTok{# Corriente es NA}
\NormalTok{      Corriente <-}\StringTok{ }\OtherTok{NA}
      
      \CommentTok{# El transecto esta determinado por el ciclo}
\NormalTok{      Transecto <-}\StringTok{ }\NormalTok{k}
      
      \CommentTok{# Obtener un numero aleatorio para la riqueza}
\NormalTok{      n_spp <-}\StringTok{ }\KeywordTok{runif}\NormalTok{(}\DataTypeTok{n =} \DecValTok{1}\NormalTok{, }\DataTypeTok{min =} \DecValTok{0}\NormalTok{, }\DataTypeTok{max =} \DecValTok{10}\NormalTok{) }\OperatorTok\StringTok{ }
\StringTok{        }\KeywordTok{as.integer}\NormalTok{()}
      
      \CommentTok{# Muestrear la lista de especies para obtener las}
      \CommentTok{# observadas en este transecto}
\NormalTok{      GeneroEspecie <-}\StringTok{ }\KeywordTok{sample}\NormalTok{(genero_especie,}
                              \DataTypeTok{size =}\NormalTok{ n_spp)}
      \CommentTok{# Sexo es NA}
\NormalTok{      Sexo <-}\StringTok{ }\OtherTok{NA}
      
      \CommentTok{# La funcion rc me dice los pares RC}
\NormalTok{      RC <-}\StringTok{ }\KeywordTok{rc}\NormalTok{(Sitio)}
      
      \CommentTok{# Definir un ciclo para iterar cada especie}
      \ControlFlowTok{for}\NormalTok{(l }\ControlFlowTok{in}\NormalTok{ GeneroEspecie)\{}
        
        \CommentTok{# Separar genero y especie}
\NormalTok{        Genero <-}\StringTok{ }\KeywordTok{str_split}\NormalTok{(l, }\StringTok{" "}\NormalTok{)[[}\DecValTok{1}\NormalTok{]][[}\DecValTok{1}\NormalTok{]]}
\NormalTok{        Especie <-}\StringTok{ }\KeywordTok{str_split}\NormalTok{(l, }\StringTok{" "}\NormalTok{)[[}\DecValTok{1}\NormalTok{]][[}\DecValTok{2}\NormalTok{]]}
        
        \CommentTok{# Obtener un numero aleatorio entre 1 y 5 para}
        \CommentTok{# definir el numero de grupos de tallas observados}
\NormalTok{        nobs <-}\StringTok{ }\KeywordTok{sample}\NormalTok{(}\DataTypeTok{x =} \DecValTok{1}\OperatorTok{:}\DecValTok{5}\NormalTok{, }\DataTypeTok{size =}\NormalTok{ 1L)}
        
        \CommentTok{# Definir un ciclo para iterar cada grupo de observaciones de una spp}
        \ControlFlowTok{for}\NormalTok{(m }\ControlFlowTok{in} \DecValTok{1}\OperatorTok{:}\NormalTok{nobs)\{}
          
          \CommentTok{# Escoger una talla aleatoria segun la funcion anterior}
\NormalTok{          Talla <-}\StringTok{ }\KeywordTok{tallas}\NormalTok{(}\DataTypeTok{spp =}\NormalTok{ spp, }\DataTypeTok{generoespecie =}\NormalTok{ l)}
          
          \CommentTok{# Clase talla es constante (NA)}
\NormalTok{          ClaseTalla <-}\StringTok{ }\OtherTok{NA}
          
          \CommentTok{# Muestrear una abundanciasegun la funcion}
\NormalTok{          Abundancia <-}\StringTok{ }\KeywordTok{mean_sp}\NormalTok{(}\DataTypeTok{generoespecie =}\NormalTok{ l)}
          
          \CommentTok{# Juntar las observaciones de este grupo de tallas}
\NormalTok{          datos_ijklm <-}\StringTok{ }\KeywordTok{tibble}\NormalTok{(Dia,}
\NormalTok{                                Mes,}
\NormalTok{                                Ano,}
\NormalTok{                                Estado,}
\NormalTok{                                Comunidad,}
\NormalTok{                                Sitio,}
\NormalTok{                                Latitud,}
\NormalTok{                                Longitud,}
\NormalTok{                                Habitat,}
\NormalTok{                                Zona,}
\NormalTok{                                TipoProteccion,}
\NormalTok{                                ANP,}
\NormalTok{                                BuzoMonitor, }
\NormalTok{                                HoraInicial,}
\NormalTok{                                ProfundidadInicial,}
\NormalTok{                                ProfundidadFinal,}
\NormalTok{                                Temperatura,}
\NormalTok{                                Visibilidad,}
\NormalTok{                                Corriente,}
\NormalTok{                                Transecto,}
\NormalTok{                                Genero,}
\NormalTok{                                Especie,}
                                \DataTypeTok{GeneroEspecie =}\NormalTok{ l,}
\NormalTok{                                Sexo,}
\NormalTok{                                Talla,}
\NormalTok{                                ClaseTalla,}
\NormalTok{                                Abundancia,}
\NormalTok{                                RC)}
          
\NormalTok{          datos <-}\StringTok{ }\KeywordTok{rbind}\NormalTok{(datos, datos_ijklm)}
\NormalTok{        \} }\CommentTok{# Fin nobs}
\NormalTok{      \} }\CommentTok{# Fin especie}
\NormalTok{    \} }\CommentTok{# Fin transecto}
\NormalTok{  \} }\CommentTok{# Fin sitio}
\NormalTok{\} }\CommentTok{# Fin años}

\CommentTok{# Borrar los NAs originales y agrupar grupos de}
\CommentTok{# talla en caso de que esten duplicados}
\NormalTok{datos }\OperatorTok
\StringTok{  }\KeywordTok{drop_na}\NormalTok{(dia) }\OperatorTok\StringTok{ }
\StringTok{  }\KeywordTok{group_by}\NormalTok{(Dia, Mes, Ano, Estado, Comunidad, Sitio, Latitud,}
\NormalTok{           Longitud, Habitat, Zona, TipoProteccion, ANP, BuzoMonitor, }
\NormalTok{           HoraInicial, ProfundidadInicial, ProfundidadFinal,}
\NormalTok{           Temperatura, Visibilidad, Corriente, Transecto, Genero,}
\NormalTok{           Especie, GeneroEspecie, Sexo, Talla, ClaseTalla, RC) }\OperatorTok\StringTok{ }
\StringTok{  }\KeywordTok{summarize}\NormalTok{(}\DataTypeTok{Abundancia =} \KeywordTok{sum}\NormalTok{(Abundancia, }\DataTypeTok{na.rm =}\NormalTok{ T)) }\OperatorTok\StringTok{ }
\StringTok{  }\KeywordTok{ungroup}\NormalTok{() }\OperatorTok\StringTok{ }
\StringTok{  }\KeywordTok{select}\NormalTok{(Dia, Mes, Ano, Estado, Comunidad, Sitio, Latitud,}
\NormalTok{         Longitud, Habitat, Zona, TipoProteccion, ANP, BuzoMonitor, }
\NormalTok{         HoraInicial, ProfundidadInicial, ProfundidadFinal,}
\NormalTok{         Temperatura, Visibilidad, Corriente, Transecto, Genero,}
\NormalTok{         Especie, GeneroEspecie, Sexo, Talla, ClaseTalla, Abundancia, RC)}
\end{Highlighting}
\end{Shaded}

\begin{Shaded}
\begin{Highlighting}[]
\CommentTok{# Graficar los datos}
\NormalTok{datos }\OperatorTok\StringTok{ }
\StringTok{  }\KeywordTok{ggplot}\NormalTok{(}\KeywordTok{aes}\NormalTok{(}\DataTypeTok{x =}\NormalTok{ Ano, }\DataTypeTok{y =}\NormalTok{ Abundancia, }\DataTypeTok{color =}\NormalTok{ Zona, }\DataTypeTok{group =}\NormalTok{ Sitio, }\DataTypeTok{linetype =}\NormalTok{ RC)) }\OperatorTok{+}
\StringTok{  }\KeywordTok{geom_point}\NormalTok{(}\DataTypeTok{alpha =} \FloatTok{0.5}\NormalTok{, }\DataTypeTok{size =} \FloatTok{0.5}\NormalTok{) }\OperatorTok{+}
\StringTok{  }\KeywordTok{stat_summary}\NormalTok{(}\DataTypeTok{geom =} \StringTok{"line"}\NormalTok{, }\DataTypeTok{fun.y =} \StringTok{"mean"}\NormalTok{, }\DataTypeTok{size =} \DecValTok{1}\NormalTok{) }\OperatorTok{+}
\StringTok{  }\KeywordTok{facet_wrap}\NormalTok{(}\OperatorTok{~}\NormalTok{GeneroEspecie, }\DataTypeTok{ncol =} \DecValTok{3}\NormalTok{, }\DataTypeTok{scales =} \StringTok{"free_y"}\NormalTok{) }\OperatorTok{+}
\StringTok{  }\NormalTok{startR}\OperatorTok{::}\KeywordTok{ggtheme_plot}\NormalTok{() }\OperatorTok{+}
\StringTok{  }\KeywordTok{theme}\NormalTok{(}\DataTypeTok{legend.position =} \StringTok{"top"}\NormalTok{) }\OperatorTok{+}
\StringTok{  }\KeywordTok{scale_color_brewer}\NormalTok{(}\DataTypeTok{palette =} \StringTok{"Set1"}\NormalTok{) }\OperatorTok{+}
\StringTok{  }\KeywordTok{xlab}\NormalTok{(}\StringTok{"Año"}\NormalTok{)}
\end{Highlighting}
\end{Shaded}

\begin{figure}
\centering
\includegraphics{evaluacion-reservas_files/figure-latex/graficar-datos-1.pdf}
\caption{\label{fig:graficar-datos}Series de tiempo de los datos antes de
agregar tendencias}
\end{figure}

\begin{Shaded}
\begin{Highlighting}[]
\CommentTok{# Ahora agregamos tendencias en abundancias y tallas.}
\CommentTok{# Las abundancias aumentan un 10% cada ano despues del 2000.}
\CommentTok{# Las tallas aumentan 1 cm cada año.}
\NormalTok{datos <-}\StringTok{ }\NormalTok{datos }\OperatorTok
\StringTok{  }\KeywordTok{mutate}\NormalTok{(}\DataTypeTok{neg =} \KeywordTok{ifelse}\NormalTok{(RC }\OperatorTok{==}\StringTok{ "Cerro prieto - Popotla"}\NormalTok{, }\DecValTok{-1}\NormalTok{, }\DecValTok{1}\NormalTok{),}
         \DataTypeTok{Abundancia =} \KeywordTok{ifelse}\NormalTok{(Zona }\OperatorTok{==}\StringTok{ "Reserva"} \OperatorTok{&}\StringTok{ }\NormalTok{Ano }\OperatorTok{>}\StringTok{ }\DecValTok{2000}\NormalTok{,}
\NormalTok{                             Abundancia }\OperatorTok{*}\StringTok{ }\NormalTok{(}\DecValTok{1} \OperatorTok{+}\StringTok{ }\NormalTok{(neg }\OperatorTok{*}\StringTok{ }\NormalTok{((Ano }\OperatorTok{-}\StringTok{ }\DecValTok{2000}\NormalTok{) }\OperatorTok{*}\StringTok{ }\FloatTok{0.1}\NormalTok{))),}
\NormalTok{                             Abundancia),}
         \DataTypeTok{Talla =} \KeywordTok{ifelse}\NormalTok{(Zona }\OperatorTok{==}\StringTok{ "Reserva"} \OperatorTok{&}\StringTok{ }\NormalTok{Ano }\OperatorTok{>}\StringTok{ }\DecValTok{2000}\NormalTok{,}
\NormalTok{                        Talla }\OperatorTok{+}\StringTok{ }\NormalTok{((Ano }\OperatorTok{-}\StringTok{ }\DecValTok{2000}\NormalTok{) }\OperatorTok{*}\StringTok{ }\DecValTok{1}\NormalTok{),}
\NormalTok{                        Talla)) }\OperatorTok\StringTok{ }
\StringTok{  }\KeywordTok{select}\NormalTok{(}\OperatorTok{-}\NormalTok{neg)}
\end{Highlighting}
\end{Shaded}

\begin{Shaded}
\begin{Highlighting}[]
\CommentTok{# Graficar los datos}
\NormalTok{datos }\OperatorTok\StringTok{ }
\StringTok{  }\KeywordTok{ggplot}\NormalTok{(}\KeywordTok{aes}\NormalTok{(}\DataTypeTok{x =}\NormalTok{ Ano, }\DataTypeTok{y =}\NormalTok{ Abundancia, }\DataTypeTok{color =}\NormalTok{ Zona, }\DataTypeTok{group =}\NormalTok{ Sitio, }\DataTypeTok{linetype =}\NormalTok{ RC)) }\OperatorTok{+}
\StringTok{  }\KeywordTok{geom_point}\NormalTok{(}\DataTypeTok{alpha =} \FloatTok{0.5}\NormalTok{, }\DataTypeTok{size =} \FloatTok{0.5}\NormalTok{) }\OperatorTok{+}
\StringTok{  }\KeywordTok{stat_summary}\NormalTok{(}\DataTypeTok{geom =} \StringTok{"line"}\NormalTok{, }\DataTypeTok{fun.y =} \StringTok{"mean"}\NormalTok{, }\DataTypeTok{size =} \DecValTok{1}\NormalTok{) }\OperatorTok{+}
\StringTok{  }\KeywordTok{facet_wrap}\NormalTok{(}\OperatorTok{~}\NormalTok{GeneroEspecie, }\DataTypeTok{ncol =} \DecValTok{3}\NormalTok{, }\DataTypeTok{scales =} \StringTok{"free_y"}\NormalTok{) }\OperatorTok{+}
\StringTok{  }\NormalTok{startR}\OperatorTok{::}\KeywordTok{ggtheme_plot}\NormalTok{() }\OperatorTok{+}
\StringTok{  }\KeywordTok{theme}\NormalTok{(}\DataTypeTok{legend.position =} \StringTok{"top"}\NormalTok{) }\OperatorTok{+}
\StringTok{  }\KeywordTok{scale_color_brewer}\NormalTok{(}\DataTypeTok{palette =} \StringTok{"Set1"}\NormalTok{) }\OperatorTok{+}
\StringTok{  }\KeywordTok{xlab}\NormalTok{(}\StringTok{"Año"}\NormalTok{)}
\end{Highlighting}
\end{Shaded}

\begin{figure}
\centering
\includegraphics{evaluacion-reservas_files/figure-latex/graficar-datos-abundancias-1.pdf}
\caption{\label{fig:graficar-datos-abundancias}Series de tiempo de los datos
con tendencias (10\% anual) despés del primer año. Note como una reserva
funciona y otra no.}
\end{figure}

\begin{Shaded}
\begin{Highlighting}[]
\CommentTok{# Graficar los datos}
\NormalTok{datos }\OperatorTok\StringTok{ }
\StringTok{  }\KeywordTok{ggplot}\NormalTok{(}\KeywordTok{aes}\NormalTok{(}\DataTypeTok{x =}\NormalTok{ Ano, }\DataTypeTok{y =}\NormalTok{ Talla, }\DataTypeTok{color =}\NormalTok{ Zona, }\DataTypeTok{group =}\NormalTok{ Sitio, }\DataTypeTok{linetype =}\NormalTok{ RC)) }\OperatorTok{+}
\StringTok{  }\KeywordTok{geom_point}\NormalTok{(}\DataTypeTok{alpha =} \FloatTok{0.5}\NormalTok{, }\DataTypeTok{size =} \FloatTok{0.5}\NormalTok{) }\OperatorTok{+}
\StringTok{  }\KeywordTok{stat_summary}\NormalTok{(}\DataTypeTok{geom =} \StringTok{"line"}\NormalTok{, }\DataTypeTok{fun.y =} \StringTok{"mean"}\NormalTok{, }\DataTypeTok{size =} \DecValTok{1}\NormalTok{) }\OperatorTok{+}
\StringTok{  }\KeywordTok{facet_wrap}\NormalTok{(}\OperatorTok{~}\NormalTok{GeneroEspecie, }\DataTypeTok{ncol =} \DecValTok{3}\NormalTok{, }\DataTypeTok{scales =} \StringTok{"free_y"}\NormalTok{) }\OperatorTok{+}
\StringTok{  }\NormalTok{startR}\OperatorTok{::}\KeywordTok{ggtheme_plot}\NormalTok{() }\OperatorTok{+}
\StringTok{  }\KeywordTok{theme}\NormalTok{(}\DataTypeTok{legend.position =} \StringTok{"top"}\NormalTok{) }\OperatorTok{+}
\StringTok{  }\KeywordTok{scale_color_brewer}\NormalTok{(}\DataTypeTok{palette =} \StringTok{"Set1"}\NormalTok{) }\OperatorTok{+}
\StringTok{  }\KeywordTok{xlab}\NormalTok{(}\StringTok{"Año"}\NormalTok{)}
\end{Highlighting}
\end{Shaded}

\begin{figure}
\centering
\includegraphics{evaluacion-reservas_files/figure-latex/graficar-datos-tallas-1.pdf}
\caption{\label{fig:graficar-datos-tallas}Series de tiempo de los datos con
tendencias (10\% anual) despés del primer año. Note como una reserva
funciona y otra no.}
\end{figure}

\begin{Shaded}
\begin{Highlighting}[]
\KeywordTok{write.csv}\NormalTok{(}\DataTypeTok{x =}\NormalTok{ datos,}
          \DataTypeTok{file =}\NormalTok{ here}\OperatorTok{::}\KeywordTok{here}\NormalTok{(}\StringTok{"materiales"}\NormalTok{, }\StringTok{"datos"}\NormalTok{, }\StringTok{"datos_peces.csv"}\NormalTok{),}
          \DataTypeTok{row.names =}\NormalTok{ F)}
\end{Highlighting}
\end{Shaded}

\bibliography{references.bib}

\backmatter
\printindex

\end{document}
